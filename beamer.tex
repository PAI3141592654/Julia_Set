\documentclass[serif]{beamer}
\usepackage{xeCJK}
\usepackage{amsfonts} 
\usepackage{amssymb}
\newcommand\hmmax{0} 
% \newcommand\bmmax{0} 
\usepackage{bm}
\usepackage{amsthm,amsmath,amssymb}
\usepackage{mathrsfs}
\usepackage{latexsym}
\usepackage{geometry}
\usepackage{fancyhdr}
\usepackage{mathtools}
\usepackage{tikz}
\usepackage{palatino}
\usepackage{cite}
\usepackage{url}
\usepackage{listings}
\usetikzlibrary{arrows.meta,snakes,backgrounds,shapes.geometric}
\usefonttheme{professionalfonts}%
\usetheme{Berlin}
\usecolortheme{whale}
\useinnertheme[shadow=true]{rounded}
\usefonttheme[onlymath]{serif}
\setbeamercovered{transparent}
\setbeamertemplate{navigation symbols}{} % 
\setbeamertemplate{footline}{%
\leavevmode%
\hbox{%

\begin{beamercolorbox}[wd=.233333\paperwidth,ht=2.25ex,dp=1ex,center]{author in head/foot}%
\usebeamerfont{author in head/foot}\insertshortauthor
\end{beamercolorbox}%

\begin{beamercolorbox}[wd=.533333\paperwidth,ht=2.25ex,dp=1ex,center]{title in head/foot}%
\usebeamerfont{title in head/foot}\insertshorttitle
\end{beamercolorbox}%

\begin{beamercolorbox}[wd=.233333\paperwidth,ht=2.25ex,dp=1ex,right]{date in head/foot}%
\usebeamerfont{date in head/foot}\insertshortdate{}\hspace*{2em}
\insertframenumber{} / \inserttotalframenumber\hspace*{2ex}
\end{beamercolorbox}}%

\vskip0pt%
}% 

\newcommand{\song}{\CJKfamily{song}} 
\newcommand{\fs}{\CJKfamily{fs}} 
\newcommand{\kai}{\CJKfamily{kai}} 
\newcommand{\hei}{\CJKfamily{hei}} 
\newcommand{\li}{\CJKfamily{li}} 
\newcommand{\you}{\CJKfamily{you}} 
\newcommand{\chuhao}{\fontsize{42pt}{\baselineskip}\selectfont} 
\newcommand{\xiaochuhao}{\fontsize{36pt}{\baselineskip}\selectfont} 
\newcommand{\yichu}{\fontsize{32pt}{\baselineskip}\selectfont} 
\newcommand{\yihao}{\fontsize{28pt}{\baselineskip}\selectfont} 
\newcommand{\erhao}{\fontsize{21pt}{\baselineskip}\selectfont} 
\newcommand{\xiaoerhao}{\fontsize{18pt}{\baselineskip}\selectfont} 
\newcommand{\sanhao}{\fontsize{15.75pt}{\baselineskip}\selectfont} 
\newcommand{\sihao}{\fontsize{14pt}{\baselineskip}\selectfont} 
\newcommand{\xiaosihao}{\fontsize{12pt}{\baselineskip}\selectfont} 
\newcommand{\wuhao}{\fontsize{10.5pt}{\baselineskip}\selectfont} 
\newcommand{\xiaowuhao}{\fontsize{9pt}{\baselineskip}\selectfont} 
\newcommand{\liuhao}{\fontsize{7.875pt}{\baselineskip}\selectfont} 
\newcommand{\qihao}{\fontsize{5.25pt}{\baselineskip}\selectfont} 



\begin{document}

\title[title]{Julia Set的介绍及计算机绘图}

\author[]{彭湃} 

\institute[Sch]{

Hangzhou, China \\

Zhejiang University }

\date[Date below]{\today} 

%%%%%%%%%%%%%%%%%%%%%%%%%%%%%%%%%%%%%%%%%%%%%%%%%%%%%%%%%%%%%%%%

\begin{frame}

\thispagestyle{empty}

\titlepage

\end{frame}

%%%%%%%%%%%%%%%%%%%%%%%%%%%%%%%%%%%%%%%%%%%%%%%%%%%%%%%%%%%%%%%%

\begin{frame}{目录}

\tableofcontents 

\end{frame}

%%%%%%%%%%%%%%%%%%%%%%%%%%%%%%%%%%%%%%%%%%%%%%%%%%%%%%%%%%%%%%%%

\section{数学理论}

\begin{frame}{朱利亚集合的定义}

\begin{itemize}
\item 朱利亚集合有多种定义方式,为方便作图,这里采用复二项式迭代的定义

\item 朱利亚集合$Z_{c,n}$可定义为使得序列$(z, f_{c,n}(z),f_{c,n}(f_{c,n}(z)),f_{c,n}(f_{c,n}(f_{c,n}(z))), ...)$收敛的复数z的集合,其中$f_{c,n}(z)=z^n+c$,c是一个复数,n是一个正整数,不加说明时默认n=2(Normal Julia Set).

\item 下文还会考虑n>2(multi-Julia Set)的情形\cite{JuliaSet}

\end{itemize}

\end{frame}

%%%%%%%%%%%%%%%%%%%%%%%%%%%%%%%%%%%%%%%%%%%%%%%%%%%%%%%%%%%%%%%%

\begin{frame}{朱利亚集合与曼德博集合的关系}

\begin{block}{$Z_{c,2}$的连通性取决于c与曼德博集合的关系}
设曼德博集合为M,根据曼德博集合和朱利亚集合的定义,对于复平面内一点c,若$c\in M$,则对应的朱利亚集合$Z_{c,2}$在通常意义下连通;若$c\notin M$,则对应的朱利亚集合$Z_{c,2}$在通常意义下不连通。\cite{TheJuliaSet}

\begin{figure}
\includegraphics[width=4.8cm,height=2cm]{7}
\includegraphics[width=4.8cm,height=2cm]{8}
\caption{连通(左2黑色部分)和不连通(右1黑色部分)朱利亚集合及对应c(左1、右2蓝点)与曼德博集合的关系}
\end{figure}

\end{block}



\end{frame}

%%%%%%%%%%%%%%%%%%%%%%%%%%%%%%%%%%%%%%%%%%%%%%%%%%%%%%%%%%%%%%%%

\section{算法设计及算例}

%%%%%%%%%%%%%%%%%%%%%%%%%%%%%%%%%%%%%%%%%%%%%%%%%%%%%%%%%%%%%%%%

\begin{frame}{算法设计}

\begin{figure}
\includegraphics[width=6cm,height=6cm]{9}

\caption{在屏幕上建立坐标系,获取每一个像素点的坐标,设计算法判断它迭代后是否收敛。并根据迭代次数赋予像素点颜色。流程如上:}
\end{figure}

\end{frame}

\begin{frame}{算例及分析}

给定\{n,x,y\},其中$c=x+iy$,程序将逐一对每个像素点进行上述操作,得到图像。下一页展示的六个例子分别取\{2,0.1,0.62\},\{3,0.5,0.6\},\{4,0.5,0.61\},\{5,0.2,0.695\},\{6,0.51,0.61\},\\\{12,0.61245,0.61245\},其中第一张是通常意义下的朱利亚集合,而其他五张图则是n>2的情况。由图可知,将图形旋转$\frac{2\pi}{n}$后得到的图形与原图重合。在实践过程中,我们可以发现在曼德博集合的边界附近选择c的值,得到的图形细节最多(如下一页的图片)。

\end{frame}

\begin{frame}{算例及分析}
\begin{figure}
\includegraphics[width=3.6cm,height=2.4cm]{2}
\includegraphics[width=3.6cm,height=2.4cm]{3}
\includegraphics[width=3.6cm,height=2.4cm]{4}
\includegraphics[width=3.6cm,height=2.4cm]{5}
\includegraphics[width=3.6cm,height=2.4cm]{6}
\includegraphics[width=3.6cm,height=2.4cm]{12}
\end{figure}
\end{frame}

%%%%%%%%%%%%%%%%%%%%%%%%%%%%%%%%%%%%%%%%%%%%%%%%%%%%%%%%%%%%%%%%

\section{总结}

%%%%%%%%%%%%%%%%%%%%%%%%%%%%%%%%%%%%%%%%%%%%%%%%%%%%%%%%%%%%%%%%

\begin{frame}{总结及参考文献}

朱利亚集合是一种复杂分形,可以通过复二项式迭代来定义。根据这种定义,我们能用计算机作出其图形。定义式中的c与曼德博集合的关系决定了朱利亚集合的连通性。

\begin{center}

\textcolor{blue}{\Huge{THANK YOU!}}

\end{center}

\end{frame}

%%%%%%%%%%%%%%%%%%%%%%%%%%%%%%%%%%%%%%%%%%%%%%%%%%%%%%%%%%%%%%%%

\bibliographystyle{plain}
\bibliography{ref}
\end{document}
