\documentclass{ctexart}

\usepackage{graphicx}
\usepackage{amsmath}
\usepackage{listings}
\usepackage{cite}
\usepackage{url}
\usepackage{tikz}
\usepackage{amsthm}
\usepackage{palatino}

\usetikzlibrary{arrows.meta,snakes,backgrounds,shapes.geometric}

\title{Julia Set的介绍及计算机绘图}


\author{彭湃3210100061 \\ 数学与应用数学(强基计划)}


\begin{document}

\maketitle

\begin{abstract}

朱利亚集合是经典的分形之一,利用计算机可作出其近似图形。本文简单描述了朱利亚集合,说明了其与曼德博集合的关系,提供了用计算机作出其近似图形的一种方法(包括n>2的情形),并且提供了示例。

\end{abstract}

\section{引言}

\textbf{朱利亚集合}(Julia set)是一种在复平面上组成分形的点的集合,以数学家加斯顿·朱利亚(Gaston Julia)的名字命名。其通过复二项式迭代生成复杂的图案。运用多种编程语言,我们可以在计算机上作出朱利亚集合的近似图。\cite{JuliaSet}

\section{数学理论}

\subsection{定义}
朱利亚集合$Z_{c,n}$可定义为使得序列$(z, f_{c,n}(z),f_{c,n}(f_{c,n}(z)),f_{c,n}(f_{c,n}(f_{c,n}(z))), ...)$收敛的复数z的集合,其中$f_{c,n}(z)=z^n+c$,c是一个复数,n是一个正整数,不加说明时默认n=2(Normal Julia Set).\textbf{(本文还会考虑n>2(multi-Julia Set)的情形)}\cite{JuliaSet}

\subsection{与曼德博集合的关系}
设曼德博集合为M,根据曼德博集合和朱利亚集合的定义,对于复平面内一点c,若$c\in M$,则对应的朱利亚集合$Z_{c,2}$在通常意义下连通;若$c\notin M$,则对应的朱利亚集合$Z_{c,2}$在通常意义下不连通。\cite{TheJuliaSet}

\begin{figure}
\includegraphics[width=12cm,height=5cm]{7}
\includegraphics[width=12cm,height=5cm]{8}
\caption{连通(右上黑色部分)和不连通(右下黑色部分)朱利亚集合及对应c(左上、左下蓝点)与曼德博集合的关系\cite{TheJuliaSet}}
\end{figure}

\section{算法设计与算例}

\subsection{算法设计}

根据朱利亚集合的定义,我们可以在屏幕上建立坐标系,对于图片文件的每一个像素点,我们获取它的坐标,设计算法判断它是否收敛。算法的基本思路是不断迭代,每次迭代进行判断,直至中途退出$(z\notin M)$或迭代次数达到上限$(z\in M)$,根据迭代次数赋予像素点颜色。算法流程如下:


\thispagestyle{empty}
\tikzstyle{startstop} = [rectangle, rounded corners, minimum width = 2cm, minimum height=1cm,text centered, draw = black]
\tikzstyle{io} = [trapezium, trapezium left angle=70, trapezium right angle=110, minimum width=2cm, minimum height=1cm, text centered, draw=black]
\tikzstyle{process} = [rectangle, minimum width=3cm, minimum height=1cm, text centered, draw=black]
\tikzstyle{decision} = [diamond, aspect = 3, text centered, draw=black]
\tikzstyle{arrow} = [->,>=stealth]
\begin{tikzpicture}[node distance=1cm]
\node[startstop](start){Start};
\coordinate (point1) at (-5.2cm, -6cm);
\coordinate (point2) at (4.2cm, -6cm);
\coordinate (point3) at (4.2cm, -4cm);
\node[io, below of = start, yshift = -1cm](in){get coordinate $z=x+iy$};
\node[process, below of = in, yshift = -1cm](pro0){$z=z^n+c,i=i+1$};
\node[decision, below of = pro0, yshift = -1cm](dec1){$|z|>2?,i>MaxTimes?$};
\node[io, below of = dec1, yshift = -1cm](pro2){$z\notin Z_{n,c}$};
\node[io, below of = point1, yshift = -1cm](pro1){$z\in Z_{n,c}$};
\node[process, below of = pro2, yshift = -1cm](out2){Setcolor(i)};
\node[startstop, below of = out2, yshift = -1cm](stop){Stop};
\draw [arrow] (start) -- (in);
\draw [arrow] (in) -- (pro0);
\draw [arrow] (pro0) -- (dec1);
\draw (dec1) -- node [above] {No,Yes} (point1);
\draw [arrow] (point1) -- (pro1);
\draw (dec1) -- node [below] {No,No} (point2);
\draw (dec1) -- (point2) -- (point3);
\draw [arrow] (point3) -- (pro0);
\draw [arrow] (dec1) -- node [right] {Yes,--} (pro2);
\draw [arrow] (pro1) -- (out2);
\draw [arrow] (pro2) -- (out2);
\draw [arrow] (out2) -- (stop);
\end{tikzpicture}

\subsection{算例及分析}
给定\{n,x,y\},其中$c=x+iy$,程序将逐一对每个像素点进行上述操作,得到图像。下面展示的六个例子分别取\{2,0.1,0.62\},\{3,0.5,0.6\},\{4,0.5,0.61\},\\\{5,0.2,0.695\},\{6,0.51,0.61\},\{12,0.61245,0.61245\},其中第一张是通常意义下的朱利亚集合,而其他五张图则是n>2的情况。由图可知,将图形旋转$\frac{2\pi}{n}$后得到的图形与原图重合。在实践过程中,我们可以发现在曼德博集合的边界附近选择c的值,得到的图形细节最多(如下面的图片)。


\begin{figure}
\includegraphics[width=3.6cm,height=2.4cm]{2}
\includegraphics[width=3.6cm,height=2.4cm]{3}
\includegraphics[width=3.6cm,height=2.4cm]{4}
\includegraphics[width=3.6cm,height=2.4cm]{5}
\includegraphics[width=3.6cm,height=2.4cm]{6}
\includegraphics[width=3.6cm,height=2.4cm]{12}
\end{figure}

\section{总结}

朱利亚集合是一种复杂分形,可以通过复二项式迭代来定义。根据这种定义,我们能用计算机作出其图形。定义式中的c与曼德博集合的关系决定了朱利亚集合的连通性。

\bibliographystyle{plain}
\bibliography{ref}

\end{document}
